\documentclass[12 pt]{article}

% Hugo, tu peux rechanger, c'était juste pour pouvoir compiler avec pdflatex.
%\usepackage{fontspec}
%\setmainfont{Linux Libertine O}
\usepackage{titlesec}
\usepackage{lastpage}
\usepackage{totcount}
\usepackage[francais]{babel}
\usepackage[utf8]{inputenc}
\usepackage{fullpage}

\pagestyle{empty}
\parskip=8 pt
\raggedright

\titlelabel{Article \thetitle{} — }

\newcommand{\Nom}{\textsc{Nantes Machine Learning Association}}
\newcommand{\Sigle}{NMLA}

\regtotcounter{section}

\title{Statuts de l'association \\
  \Nom \\
  \small Assemblée Constitutive du 9 avril 2018}
\date{}

\begin{document}

\maketitle

\section{Constitution}
\label{sec:constitution}

Il est fondé entre les adhérents aux présents statuts une association
régie par la loi du 1\ier{} juillet 1901 modifiée par ses textes
d'application.

\section{Dénomination}
\label{sec:denomination}
L'association a pour dénomination~: \og\Nom\fg.

Elle peut être désignée par le sigle~: \og\Sigle\fg.

\section{Objet et moyens}
\label{sec:objet-et-moyens}

L'association \Nom{} a pour objet désintéressé et non lucratif~:

\begin{itemize}
\item la promotion de l'apprentissage automatique, et sa théorie et
  ses techniques, dans la région nantaise
\item l'organisation du Nantes Machine Learning Meetup (NMLM) et tout
  autre manifestation qui lui semble utile dans l'achèvement des ses objectifs
\end{itemize}

Les moyens d'action de l'association \Nom{} sont notamment la NMLM, la
NMLRG (NML Reading Group), et tout autre moyen qui semble, aux membres
de l'association et à son comité directeur, utile et raisonnable.

\section{Siège social}
\label{sec:siege-social}

Le siège de l'association se situe au 14 bis rue Metzinger 44000 Nantes.

Il pourra être transféré à tout moment par simple décision du conseil
d'administration.

\section{Durée}
\label{sec:duree}

La durée de l'association est illimitée. L'année sociale court du
1\ier{} septembre au 31 août.

\section{Cotisation annuelle}
\label{sec:cotisation-annuelle}

L'existence et le montant de la cotisation annuelle dont chaque membre
doit s'acquitter sont votés pendant l'assemblée générale.

Chaque membre doit régler sa cotisation annuelle au plus tard 3 mois
après le début de l'année sociale (la date limite de paiement est donc
le 30 novembre). Les cotisations seront réglées par l'un des moyens
suivant~:

\begin{itemize}
\item chèque libellé au nom de l'association adressé au siège de
  l'association, à l'attention du trésorier de l'association~;
\item virement bancaire sur le compte bancaire de l'association~;
\item espèces, contre remise d’un justificatif du trésorier de
  l'association.
\end{itemize}

\section{Composition de l'association}
\label{sec:composition-de-l-association}

L'association se compose de ses membres adhérents.

\section{Admission des membres}
\label{sec:admission-des-membres}

Pour être admis en tant que membre adhérent, il faut~:

\begin{itemize}
\item être parrainé par deux membres de l'association~;
\item avoir été agréé préalablement par le conseil d'administration
  qui, en cas de refus, n'aura pas à en faire connaître les raisons~;
\item formuler et signer une demande écrite~;
\item accepter intégralement les statuts et le règlement intérieur de
  l'association~;
\item s'acquitter du montant de la cotisation annuelle si l'assemblée
  générale en a fixé une, comme prévu par l'article
  \ref{sec:cotisation-annuelle}~;
\item s'engager à participer à la réalisation de l'objet de
  l'association.
\end{itemize}

Toute demande d'agrément d'un nouveau membre (accompagnée de ses
pièces justificatives) devra être adressée au bureau de l'association,
qui la transmettra après vérification de sa conformité au conseil
d'administration. Le conseil d'administration statue sur la demande
d'adhésion et décide d'agréer ou non le postulant. Le refus
d'admission n'a pas à être motivé.

\section{Perte et suspension de la qualité de membre}
\label{sec:perte-et-suspension-de-la-qualite-de-membre}

La qualité de membre de l'association se perd par :

\begin{itemize}
\item démission écrite~;
\item décès~;
\item exclusion prononcée par le conseil d'administration pour les
  motifs suivants~:
  \begin{itemize}
  \item non paiement de la cotisation votée par l'assemblée générale
    trois mois après l'échéance de celle-ci~;
  \item inactivité, définie comme l'absence d'implication dans
    l'association pendant une durée d'un an~;
  \item tout autre motif grave laissé à son appréciation. Est
    considéré comme motif grave toute initiative directe ou indirecte
    d'un membre visant à diffamer l'association ou certains de ses
    membres ou à porter atteinte aux objectifs poursuivis par
    l'association.
  \end{itemize}
\end{itemize}

S'il le juge opportun, le conseil d'administration peut décider, pour
les mêmes motifs que ceux indiqués ci-dessus, la suspension temporaire
d'un membre plutôt que son exclusion.  Cette décision implique la
perte de la qualité de membre et du droit de participer à la vie
sociale, pendant toute la durée de la suspension, telle que déterminée
par le conseil d'administration dans sa décision. Si le membre
suspendu est investi de fonctions électives, la suspension entraîne
également la cessation de son mandat.

\section{Administration}
\label{sec:administration}

Le conseil d'administration choisit parmi ses membres un bureau
composé de~:

\begin{itemize}
\item un président~;
\item un ou plusieurs vice-présidents, s'il y a lieu~;
\item un secrétaire~;
\item un ou plusieurs vice-secrétaires, s'il y a lieu~;
\item un trésorier~;
\item un ou plusieurs vice-trésoriers, s'il y a lieu.
\end{itemize}

Le bureau est élu pour un an et peut être reconduit. Le même membre
peut occuper plusieurs postes parmi les postes détaillés ci-dessus à
l'exception des postes de président et trésorier, qui ne peuvent être
simultanément occupé par la même personne.

\section{Réunion du bureau}
\label{sec:reunion-du-bureau}

Le bureau se réunit aussi souvent que l'exige l'intérêt de
l'association.

\section{Réunion du conseil d'administration}
\label{sec:reunion-du-conseil-d-administration}

Le conseil d'administration se réunit sur convocation de son Président
ou sur la demande du quart de ses membres ou aussi souvent que l'exige
l'intérêt de l'association. La présence de la moitié des membres du
conseil d'administration est nécessaire pour la validité des
délibérations. Si le quorum n'est pas atteint lors de la réunion du
conseil d'administration, ce dernier sera convoqué à nouveau à une
semaine d'intervalle, et il pourra valablement délibérer, quels que
soient le nombre de membres présents.

Les décisions sont prises à la majorité absolue des membres présents
ou représentés et les membres qui s'abstiennent lors du vote sont
considérés comme repoussant les résolutions mises au vote.

En cas de partage, la voix du président de l'association est
prépondérante.

Tout membre du conseil d'administration, qui, sans excuse, n'aura pas
assisté à trois réunions consécutives pourra être considéré comme
démissionnaire.

Il est tenu procès-verbal des séances. Les procès-verbaux sont rédigés
par le secrétaire et signés par le président. Ils sont transcrits sur
un registre coté et paraphés par le président.

\section{Pouvoir}
\label{sec:pouvoir}

Le conseil d'administration est investi des pouvoirs les plus étendus
pour faire ou autoriser tous les actes ou opérations dans la limite de
son objet et qui ne sont pas du ressort de l'assemblée générale.

Il autorise le président à agir en justice.

Il surveille la gestion des membres du bureau et a le droit de se
faire rendre compte de leurs actes.

Il arrête le budget et les comptes annuels de l'association.

Cette énumération n'est pas limitative.

Il peut faire toute délégation de pouvoirs pour une question
déterminée et un temps limité.

\section{Rôle des membres du bureau}
\label{sec:role-des-membres-du-bureau}

\paragraph{Rôle du président}

Le président convoque les assemblées générales et les réunions du
conseil d'administration. Il représente l'association dans tous les
actes de la vie civile et est investi de tous les pouvoirs à cet
effet. Il peut déléguer certaines de ses attributions.

Il a notamment qualité pour ester en justice au nom de l'association,
tant en demande qu'en défense. En cas d'absence ou de maladie, il est
remplacé par tout autre administrateur spécialement délégué par le
conseil.

\paragraph{Rôle du secrétaire}

Le secrétaire est chargé de tout ce qui concerne la correspondance et
les archives.  Il rédige les procès-verbaux des délibérations et en
assure la transcription sur les registres.  Il tient le registre
spécial, prévu par la loi, et assure l'exécution des formalités
prescrites.

\paragraph{Rôle du trésorier}

Le trésorier est chargé de tout ce qui concerne la gestion du
patrimoine de l'association. Il effectue tous paiements et perçoit
toutes recettes sous la surveillance du président.  Il tient une
comptabilité régulière, au jour le jour, de toutes les opérations et
rend compte à l'assemblée annuelle, qui statue sur la
gestion.

Toutefois, les dépenses supérieures à 200 euros doivent être
ordonnancées par le président ou, à défaut, en cas d'empêchement, par
tout autre membre du bureau. Il rend compte de son mandat aux
assemblées générales.

\section{Assemblée générale}
\label{sec:assemblee-generale}

L'assemblée générale comprend tous les membres.

Elle se réunit au moins une fois par an dans les six mois de la
clôture de l'exercice et chaque fois qu'elle est convoquée par le
conseil d'administration ou sur la demande d'au moins le quart des
membres.

L'ordre du jour est réglé par le conseil d'administration.

Le bureau de l'assemblée générale est celui du conseil d'administration.

Le président de l'association préside l'assemblée, expose la situation
morale de l'association et rend compte de l'activité de
l'association. Le trésorier rend compte de sa gestion et soumet le
bilan à l'approbation de l'assemblée générale.

L'assemblée générale délibère sur les rapports de gestion du conseil
d'administration et de situation morale et financière de
l'association. Elle approuve les comptes de l'exercice clos, vote le
budget de l'exercice suivant, décide de l'existence ou non d'une
cotisation dont doit s'acquitter chaque membre de l'association et en
fixe le montant le cas échéant et délibère sur les seules questions
inscrites à l'ordre du jour.

Elle procède à l'élection des nouveaux membres du conseil
d'administration et ratifie les nominations effectuées à titre
provisoire.

Elle autorise la conclusion des actes ou opérations qui excèdent les
pouvoirs du Conseil. En outre, elle délibère sur toutes les questions
portées à l'ordre du jour à la demande signée du tiers des membres de
l'association déposée au secrétariat dix jours au moins avant la
réunion.

Les membres convoqués régulièrement peuvent être représentés par un
autre membre par procuration écrite et signée.

Les convocations sont envoyées par courrier électronique au moins
quinze jours avant la date fixée pour la réunion et indiquent l'ordre
du jour arrêté par le conseil d'administration. Une feuille de
présence sera émargée par chaque participant et certifiée par le
bureau.

Les décisions en assemblée générale sont prises à main levée à la
majorité relative des membres présents ou représentés et les
membres qui s'abstiennent lors du vote sont considérés comme
repoussant les résolutions mises au vote. % ??

Le scrutin secret peut être demandé soit par le conseil
d'administration, soit par au moins quatre membres présents. Le
bulletin secret est obligatoire lors des votes sur les personnes.

\section{Assemblée générale extraordinaire}
\label{sec:assemblee-generale-extraordinaire}

L'assemblée générale extraordinaire est seule compétente pour modifier
les statuts, prononcer la dissolution de l'association et statuer sur
la dévolution de ses biens, décider de sa fusion avec d'autres
associations ou sa transformation.

Les modalités de décision sont les mêmes que celles prévues par
l'article \ref{sec:assemblee-generale} pour les assemblées générales.

Une feuille de présence sera émargée et certifiée par les membres du bureau.

Si le quorum n'est pas atteint lors de la réunion de l'assemblée, sur
première convocation, l'assemblée sera convoquée à nouveau à quinze
jours d'intervalle et, lors de cette nouvelle réunion, elle pourra
valablement délibérer quel que soit le nombre de membres présents ou
représentés.

\section{Procès-verbaux des assemblées générales}
\label{sec:proces-verbaux-des-assemblees-generales}

Les délibérations des assemblées sont constatées sur des
procès-verbaux contenant le résumé des débats, le texte des
délibérations et le résultat des votes.

Les procès-verbaux sont retranscrits, sans blanc ni rature, dans
l'ordre chronologique sur le registre des délibérations de
l'association, préalablement coté et paraphé par le président de
l'association.

Les procès-verbaux des délibérations sont rédigés par le secrétaire et
signés par le président et un autre membre du conseil.

Le secrétaire peut délivrer toutes copies certifiées conformes qui
font foi vis-à-vis des tiers.

\section{Dissolution}
\label{sec:dissolution}

La dissolution de l'association ne peut être prononcée que par
l'assemblée générale extraordinaire, convoquée spécialement à cet
effet et statuant aux conditions de quorum et de majorité prévues à
l'article \ref{sec:assemblee-generale-extraordinaire}.

L'assemblée générale extraordinaire désigne un ou plusieurs
liquidateurs chargés des opérations de liquidation.

Lors de la clôture de la liquidation, l'assemblée générale
extraordinaire se prononce sur la dévolution de l'actif net au profit
de toutes associations déclarées de son choix, ayant un objet
similaire.

\section{Ressources}
\label{sec:ressources}

Les ressources de l'association sont toutes celles qui ne sont pas
interdites par les lois et règlements en vigueur.

\section{Règlement intérieur}
\label{sec:reglement-interieur}

Le conseil d'administration pourra, s'il le juge nécessaire, arrêter
le texte d'un règlement intérieur, qui détermine les détails
d'exécution des présents statuts.

Ce règlement sera soumis à l'approbation de l'assemblée générale,
ainsi que ses modifications éventuelles.

\section{Formalités}
\label{sec:formalites}

Le Président, au nom du conseil d'administration, est chargé de remplir toutes formalités de déclarations et publications prescrites par le législateur.

\section*{}

Ce document relatif aux statuts de l'association \Nom{} comporte
\pageref{LastPage} pages, ainsi que \total{section} articles.

% Cf. également https://www.alchemistowl.org/pocorgtfo/pocorgtfo14.pdf
% Voir le MD5 en bas de la première page, puis la discussion qui
% commence sur la page 46 ff (14:09 ff).

\end{document}

%%% Local Variables:
%%% mode: latex
%%% TeX-master: t
%%% End:
